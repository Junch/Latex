\documentclass{ctexart}
\usepackage[cache=false]{minted}
\usepackage{bm}
\usepackage{geometry}
\geometry{a4paper,left=2.5cm,right=2cm,top=1.5cm,bottom=1.0cm}
% \newcommand{\classname}[1]{\texttt{#1}}
% \newcommand{\classname}[1]{\mintinline{cpp}{#1}}

\usepackage{xcolor}
\definecolor{light-gray}{gray}{0.85}

\definecolor{bg}{rgb}{0.95,0.95,0.95}

\newcommand{\classname}[1]{%
\colorbox{light-gray}{%
\parbox[b][0.6em]{\widthof{\mintinline{cpp}{#1}}}{\mintinline{cpp}{#1}}%
}%
}

\usepackage{hyperref}
\hypersetup{
    colorlinks,
    linkcolor={red!50!black},
    citecolor={blue!50!black},
    urlcolor={blue!80!black}
}
% \usepackage{xeCJK}
% \setCJKmainfont[BoldFont={Adobe Heiti Std}, ItalicFont={Adobe Kaiti Std}]{Adobe Song Std} % 主字体是Adobe宋



\begin{document}

Given an array nums of distinct integers, return all the possible permutations. You can return the answer in {\bfseries any order}.

\begin{minted}[escapeinside=||]{text}
|\bf{Input}|: nums = [1,2,3]
|\bf{Output}|: [[1,2,3],[1,3,2],[2,1,3],[2,3,1],[3,1,2],[3,2,1]]
\end{minted}
\vspace{0.5cm}

\centerline{\bfseries \href{https://leetcode.com/problems/permutations}{LeetCode 46. Permutations}}
\begin{minted} [frame=lines,framesep=2mm,baselinestretch=1.2,linenos]{cpp}
vector<vector<int>> permute(vector<int>& nums) {
    vector<vector<int>> ans;
    vector<int> path;
    vector<bool> used(nums.size(), false);

    function<void(void)> dfs = [&]() {
        if (path.size() == nums.size()) {
            ans.push_back(path);
            return;
        }

        for (int i = 0; i < nums.size(); ++i) {
            if (used[i])
                continue;

            path.push_back(nums[i]);
            used[i] = true;
            dfs();
            path.pop_back();
            used[i] = false;
        }
    };

    dfs();
    return ans;
}
\end{minted}

\clearpage

Given a collection of numbers, nums, that might contain duplicates, return all possible unique permutations in {\bfseries any order}.

\begin{minted}[escapeinside=||]{text}
|\bf{Input}|: nums = [1,1,2]
|\bf{Output}|:
[[1,1,2],
 [1,2,1],
 [2,1,1]]
\end{minted} 

\vspace{0.5cm}

\centerline{\bfseries \href{https://leetcode.com/problems/permutations-ii}{LeetCode 47. Permutations II}}
\begin{minted} [frame=lines,framesep=2mm,baselinestretch=1.2,linenos]{cpp}
vector<vector<int>> permuteUnique(vector<int>& nums) {
    // https://zxi.mytechroad.com/blog/searching/leetcode-47-permutations-ii/
    
    vector<vector<int>> ans;
    sort(nums.begin(), nums.end());
    vector<int> path;
    vector<bool> used(nums.size(), false);

    function<void(void)> dfs = [&]() {
        if (path.size() == nums.size()) {
            ans.push_back(path);
            return;
        }

        for (int i = 0; i < nums.size(); ++i) {
            if (used[i])
                continue;
            if (i > 0 && nums[i] == nums[i - 1] && !used[i - 1])
                continue;

            path.push_back(nums[i]);
            used[i] = true;
            dfs();
            path.pop_back();
            used[i] = false;
        }
    };

    dfs();
    return ans;
}
\end{minted}

\clearpage

Design a data structure that follows the constraints of a Least Recently Used (LRU) cache.

Implement the LRUCache class:

\begin{itemize}
    \item \classname{LRUCache(int capacity)} Initialize the LRU cache with positive size \classname{capacity}.
    \item \classname{int get(int key)} Return the value of the \classname{key} if the key exists, otherwise return \classname{-1}.
    \item \classname{void put(int key, int value)} Update the value of the \classname{key} if the key exists. Otherwise, add 
          the \classname{key-value} pair to the cache. If the number of keys exceeds the \classname{capacity} from this operation, evict the least recently used key.
    \item The functions \classname{get} and \classname{put} must each run in \texttt{O(1)} average time complexity.
\end{itemize}
\vspace{0.5cm}

\centerline{\bfseries LeetCode \href{https://leetcode.com/problems/lru-cache}{146. LRU Cache O(1)}}
\begin{minted} [frame=lines,framesep=2mm,baselinestretch=1.2,linenos]{cpp}
class LRUCache {
public:
    LRUCache(int capacity) {
        capacity_ = capacity;
    }
    
    int get(int key) {
        const auto it = m_.find(key);
        // If key does not exist
        if (it == m_.cend()) return -1;
 
        // Move this key to the front of the cache
        cache_.splice(cache_.begin(), cache_, it->second);
        return it->second->second;
    }
    
    void put(int key, int value) {        
        const auto it = m_.find(key);
        
        // Key already exists
        if (it != m_.cend()) {
            // Update the value
            it->second->second = value;
            // Move this entry to the front of the cache
            cache_.splice(cache_.begin(), cache_, it->second);
            return;
        }
        
        // Reached the capacity, remove the oldest entry
        if (cache_.size() == capacity_) {
            const auto& node = cache_.back();
            m_.erase(node.first);
            cache_.pop_back();
        }
        
        // Insert the entry to the front of the cache and update mapping.
        cache_.emplace_front(key, value);
        m_[key] = cache_.begin();
    }
private:
    int capacity_;
    list<pair<int,int>> cache_;
    unordered_map<int, list<pair<int,int>>::iterator> m_;
};
 
/**
 * Your LRUCache object will be instantiated and called as such:
 * LRUCache obj = new LRUCache(capacity);
 * int param_1 = obj.get(key);
 * obj.put(key,value);
 */
\end{minted}

\clearpage

\centerline{\bfseries \href{https://leetcode.com/problems/sort-an-array}{LeetCode 912. Sort an Array}}
Refer to \url{https://zxi.mytechroad.com/blog/algorithms/array/leetcode-912-sort-an-array}

\begin{minted} [frame=lines,framesep=2mm,baselinestretch=1.2,linenos]{cpp}
class Solution { // QuickSort
public:
    vector<int> sortArray(vector<int>& nums)
    {
        function<void(int, int)> quickSort = [&](int l, int r) {
            if (l >= r) return;
            int i = l;
            int j = r;
            int p = nums[l + rand() % (r - l + 1)];
            while (i <= j) {
                while (nums[i] < p) ++i;
                while (nums[j] > p) --j;
                if (i <= j)
                    swap(nums[i++], nums[j--]);
            }
            quickSort(l, j);
            quickSort(i, r);
        };

        quickSort(0, nums.size() - 1);
        return nums;
    }
};

class Solution { // HeapSort
public:
    vector<int> sortArray(vector<int>& nums) 
    {
        priority_queue<int> q(begin(nums), end(nums));
        int i = nums.size();
        while (!q.empty()) {
            nums[--i] = q.top();
            q.pop();
        }
        return nums;
    }
};

class Solution { // HeapSort
public:
    vector<int> sortArray(vector<int>& nums)
    {
        auto heapify = [&](int i, int e) {
            while (i <= e) {
                int l = 2 * i + 1;
                int r = 2 * i + 2;
                int j = i;
                if (l <= e && nums[l] > nums[j])
                    j = l;
                if (r <= e && nums[r] > nums[j])
                    j = r;
                if (j == i)
                    break;
                swap(nums[i], nums[j]);
                swap(i, j);
            }
        };

        const int n = nums.size();

        // Init heap
        for (int i = n / 2; i >= 0; i--)
            heapify(i, n - 1);

        // Get min.
        for (int i = n - 1; i >= 1; i--) {
            swap(nums[0], nums[i]);
            heapify(0, i - 1);
        }

        return nums;
    }
};

class Solution { // MergeSort
public:
    vector<int> sortArray(vector<int>& nums)
    {
        vector<int> t(nums.size());
        function<void(int, int)> mergeSort = [&](int l, int r) {
            if (l + 1 >= r)
                return;
            int m = l + (r - l) / 2;
            mergeSort(l, m);
            mergeSort(m, r);
            int i1 = l;
            int i2 = m;
            int index = 0;
            while (i1 < m || i2 < r)
                if (i2 == r || (i1 < m && nums[i1] < nums[i2]))
                    t[index++] = nums[i1++];
                else
                    t[index++] = nums[i2++];
            std::copy(begin(t), begin(t) + index, begin(nums) + l);
        };

        mergeSort(0, nums.size());
        return nums;
    }
};
\end{minted}

\clearpage

151. 翻转字符串里的单词: \href{https://leetcode.com/problems/reverse-words-in-a-string}{https://leetcode.com/problems/reverse-words-in-a-string}

\begin{minted} [frame=lines,framesep=2mm,baselinestretch=1.2,linenos]{cpp}
class Solution {
    public:
        void reverseWords(string& s)
        {
            int storeIndex = 0, n = s.size();
            reverse(s.begin(), s.end());
            for (int i = 0; i < n; ++i) {
                if (s[i] != ' ') {
                    if (storeIndex != 0)
                        s[storeIndex++] = ' ';
                    int j = i;
                    while (j < n && s[j] != ' ')
                        s[storeIndex++] = s[j++];
                    reverse(s.begin() + storeIndex - (j - i), s.begin() + storeIndex);
                    i = j;
                }
            }
            s.resize(storeIndex);
        }
    };
}
\end{minted}

\end{document}
