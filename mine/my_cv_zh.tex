\documentclass[11pt,a4paper,sans]{moderncv}

% moderncv 主题
\moderncvstyle{classic}
\moderncvcolor{blue}

% 字符编码
\usepackage[noindent,UTF8]{ctex}

% 调整页面
\usepackage{geometry}
\geometry{left=2cm,right=2cm,top=2cm,bottom=1cm}  %设置 上、左、下、右 页边距
%\usepackage[scale=0.75]{geometry}
%\setlength{\hintscolumnwidth}{3cm}

% Refine quote 
\newcommand*{\myquote}[2]{%
   \quote{\itshape #1 \\ \scshape \footnotesize #2}}

% 个人信息
\firstname{周}
\familyname{瑜}
\title{简历}
\myquote{Write clean code}{}
\mobile{+86~13361970595}
\email{junc76@gmail.com}
\homepage{junch.github.io}
\nopagenumbers{}  

%----------------------------------------------------------------------------------
%            内容
%----------------------------------------------------------------------------------
\begin{document}
\maketitle

\section{工作背景}
\cventry{2005.6-现在}{Autodesk公司}{PSEB部门}{高级软件开发工程师}{}{}
\cventry{\footnotesize{2003.10-2004.3}}{华为技术有限公司}{BMS固网网管部IP项目组}{软件开发工程师}{}{}

\section{项目经历}
\cventry{2013.4-至今}{AutoCAD}{In cavas UI}{C++}{}{
\begin{enumerate}
\item LMG-156: User sees refreshed cursor badge/replacement when using ZOOM/3DZOOM command
\item LMG-157: User sees cursor been replaced when using PAN/3DPAN command
\item LMG-457: CommandPreview Benchmark
\item LMG-464: when creating a line/Mline/Pline/Spline, user can preview the exact properties during creation. (Part 
of the work)
\end{enumerate}
}

\cventry{2011.4-2012.11}{AutoCAD}{Visual style}{C++ WPF}{}{
\begin{enumerate}
\item Maintain the predefined visual styles, such as adding new visual styles and updating the property values.
\item Maintain the visual style swatchs on VSM.
\item Add new ribbon controls related to visual styles.
\item Add new controls for the visual styles on the tool palette.
\item Fix bugs related to VSM, roundtrip and per-obj visual style.
\item In-canvas UI enhancement: Cursor badge, pan/3dpan and zoom/3dzoom.
\end{enumerate}
}

\vspace*{0.2\baselineskip}

\cventry{2006.10-2010.11}{AutoCAD MEP}{Tool based UI}{C++ MFC COM/ATL}{}{将传统对话框界面改成基于Palellate的界面,代码大部分重构。使用了 MFC,COM/ATL,ActiveX 等 Windows 平台下传统的组件开发技术。担任技术负责人。}

\vspace*{0.2\baselineskip}

\cventry{2003.10-2004.3}{网管软件N2000}{C++, ACE, SQL Server}{}{}{后台采用ACE作为底层通信模块。跨平台,同一套 C++代码可以分别在 Windows 和 Solaris上(Linux 也可以)编译链接运行;多线程;在本项目中我大约编写了 8000 行的 C++代码,处理了 160 个问题单。}

\section{教育背景}
\cventry{2002-2005}{计算机科学与技术硕士}{武汉理工大学}{}{}{}
\cventry{1993-1997}{建筑工程学士}{汕头大学}{}{}{}

\section{编程技能}
\cventry{编程语言}{ C = C++ > Python > Java = C\# }{}{}{}{}
\cventry{单元测试}{GoogleTest > Cpputest}{}{}{}{}

\end{document}
