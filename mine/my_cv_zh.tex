\documentclass[11pt,a4paper,sans]{moderncv}

% moderncv 主题
\moderncvstyle{classic}
\moderncvcolor{orange}

% 字符编码
\usepackage[noindent,UTF8]{ctex}
\usepackage{amssymb}

% 调整页面
\usepackage{geometry}
\geometry{left=2cm,right=2cm,top=2cm,bottom=1cm}  %设置 上、左、下、右 页边距
%\usepackage[scale=0.75]{geometry}
%\setlength{\hintscolumnwidth}{3cm}

% Refine quote 
%\newcommand*{\myquote}[2]{%
%   \quote{\itshape #1 \\ \scshape \footnotesize #2}}

\usepackage[originalcommands]{ragged2e}
\renewcommand*{\cvcomputer}[4]{%
  \cvdoubleitem{\textbf{#1}}{\small\raggedright#2}{\textbf{#3}}{\small\raggedright#4}}

% 个人信息
\firstname{陈}
\familyname{俊}
\title{Write clean code}
%\myquote{Write clean code}{}
\mobile{+86~13361970595}
\email{chenjun\_76@163.com}
\homepage{junch.github.io}
\photo[64pt][0pt]{portrait.png}  % ‘64pt’是图片必须压缩至的高度、‘0.4pt‘是图片边框的宽度
%\quote{\footnotesize Last updated: \today}
\nopagenumbers{}  

%----------------------------------------------------------------------------------
%            内容
%----------------------------------------------------------------------------------
\begin{document}
\maketitle

\section{\heiti 工作背景}
\cventry{\footnotesize{2005.6-至今}}{Autodesk公司}{PSEB部门}{高级软件开发工程师}{}{}
\cventry{\footnotesize{2003.10-2004.3}}{华为技术有限公司}{BMS固网网管部IP项目组}{软件开发工程师}{}{}

\vspace*{0.3\baselineskip}

\section{\heiti 项目经历}

\cventry{\footnotesize{2013.12-至今}}{AutoCAD}{开发支持}{Windbg, C++, Javascript, C\#}{}{
%\begin{enumerate}
%\item LMG-156: User sees refreshed cursor badge/replacement when using ZOOM/3DZOOM command
%\item LMG-157: User sees cursor been replaced when using PAN/3DPAN command
%\item LMG-457: CommandPreview Benchmark
%\item LMG-464: when creating a line/Mline/Pline/Spline, user can preview the exact properties during creation. (Part 
%of the work)
%\end{enumerate}
\begin{itemize}
\item 使用windbg等工具分析dump文件,帮助开发人员定位程序bug。
\item 维护CER Report网站,监控程序崩溃数据。
\item 开发CER Mailer网站,便于和用户沟通,重现Crash的操作步骤。
\item 分析AutoCAD性能Feedback报告。
\item 利用Sonar来静态分析C++, C\#代码的质量。
\item 集成其他部门的组件到AutoCAD。
\item 维护dwg/dgn图形文件相互转换的代码,加入regress测试。
\end{itemize}
}

\vspace*{0.3\baselineskip}

\cventry{\footnotesize{2013.4-2013.11}}{AutoCAD}{In canvas UI}{C++}{}{
%\begin{enumerate}
%\item LMG-156: User sees refreshed cursor badge/replacement when using ZOOM/3DZOOM command
%\item LMG-157: User sees cursor been replaced when using PAN/3DPAN command
%\item LMG-457: CommandPreview Benchmark
%\item LMG-464: when creating a line/Mline/Pline/Spline, user can preview the exact properties during creation. (Part 
%of the work)
%\end{enumerate}
更新绘图区的UI,让用户获得更佳的交互体验。
\begin{itemize}
\item 更新ZOOM/3DZOOM, PAN/3DPAN命令的Cursor Badge。
\item 创建Line/MLine/Pline/Spline时候,做到所见即所得。
\item 测试CommandPreview功能对性能的影响。
\end{itemize}
}

\vspace*{0.3\baselineskip}

\cventry{\footnotesize{2011.4-2012.11}}{AutoCAD}{Visual Style}{C++, WPF}{}{
%\begin{enumerate}
%\item Maintain the predefined visual styles, such as adding new visual styles and updating the property values.
%\item Maintain the visual style swatchs on VSM.
%\item Add new ribbon controls related to visual styles.
%\item Add new controls for the visual styles on the tool palette.
%\item Fix bugs related to VSM, roundtrip and per-obj visual style.
%\end{enumerate}
Visual Style指AutoCAD中实体的显示效果,比如Wireframe或者Realistic等。
\begin{itemize}
\item 维护预定义的Visual Styles,比如DB读写,属性更新等。
\item 更新Visual Style Manager来显示所有Visual Style的图标。
\item 修改和Visual Style Manager相关的系列Bugs。
\end{itemize}
}

\vspace*{0.3\baselineskip}

\cventry{\footnotesize{2006.4-2010.11}}{AutoCAD MEP}{Tool based UI}{C++, MFC, COM/ATL}{}
{更新MEP产品的界面。采用COM/ATL, ActiveX等技术,将基于MFC对话框的界面更新成基于COM接口的界面,保证整个AutoCAD产品界面风格的统一。担任技术负责人。}

\vspace*{0.3\baselineskip}

\cventry{\footnotesize{2003.10-2004.3}}{网管软件N2000}{VLAN}{C++, ACE, SQL Server}{}
{负责VLAN后台模块的设计维护工作,工作包括编写概要设计文档,编码,单元测试,代码检视,验证,处理问题单等。后台采用ACE作为底层通信模块,跨平台。在本项目中我大约编写了 8000 行的 C++代码,处理了 160 个问题单。}

\section{\heiti 教育背景}
\cventry{\footnotesize{2002.9-2005.4}}{计算机科学与技术硕士}{武汉理工大学}{}{}{}
\cventry{\footnotesize{1993.9-1997.7}}{建筑工程学士}{汕头大学}{}{}{}

\section{\heiti 相关技能}
%\textcolor{color1}
\cvcomputer{语言}{C/C++, Python, Objective C, C\#, Javascript, PowerShell}
           {测试}{GoogleTest, Cpputest, Mocha}
\cvcomputer{SCM}{Perforce, TFS, Git, SourceTree}
           {编辑}{XCode, VS, Sublime Text}
\cvcomputer{OS}{Windows, Mac OS X, Ubuntu}
           {调试}{VS, Windbg, DebugView, WPR/WPA}
\cvcomputer{其他}{Scrum, CMake, Markdown, Node.js, \LaTeX{}}
	  {英语}{CET-6}
\end{document}
