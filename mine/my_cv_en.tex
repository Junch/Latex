\documentclass[11pt,a4paper,sans]{moderncv}

% moderncv 主题
\moderncvstyle{classic}
\moderncvcolor{orange}

% 字符编码
%\usepackage[noindent,UTF8]{ctex}
\usepackage{amssymb}

% 调整页面
\usepackage{geometry}
\geometry{left=2cm,right=2cm,top=2cm,bottom=1cm}  %设置 上、左、下、右 页边距
%\usepackage[scale=0.75]{geometry}
%\setlength{\hintscolumnwidth}{3cm}

% Refine quote 
%\newcommand*{\myquote}[2]{%
%   \quote{\itshape #1 \\ \scshape \footnotesize #2}}

\usepackage[originalcommands]{ragged2e}
\renewcommand*{\cvcomputer}[4]{%
  \cvdoubleitem{\textbf{#1}}{\small\raggedright#2}{\textbf{#3}}{\small\raggedright#4}}

% 个人信息
\firstname{Jun}
\familyname{Chen}
\title{Write clean code}
%\myquote{Write clean code}{}
\mobile{+86~13361970595}
\email{junc76@gmail.com}
\homepage{junch.github.io}
\photo[64pt][0pt]{portrait.png}  % ‘64pt’是图片必须压缩至的高度、‘0.4pt‘是图片边框的宽度
\quote{\footnotesize Last updated: \today}
\nopagenumbers{}  

%----------------------------------------------------------------------------------
%            内容
%----------------------------------------------------------------------------------
\begin{document}
\maketitle

\section{Work Experience}
\cventry{2005.6-Now}{Autodesk, Inc}{PSEB Dept}{Senior SDE}{}{}
\cventry{\footnotesize{2003.10-2004.3}}{Huawei Technologies Co. Ltd}{BMS Dept}{SDE}{}{}

\vspace*{0.3\baselineskip}

\section{Project Experience}
\cventry{2013.4-2013.11}{AutoCAD - In cavas UI}{C++}{}{}{
%\begin{enumerate}
%\item LMG-156: User sees refreshed cursor badge/replacement when using ZOOM/3DZOOM command
%\item LMG-157: User sees cursor been replaced when using PAN/3DPAN command
%\item LMG-457: CommandPreview Benchmark
%\item LMG-464: when creating a line/Mline/Pline/Spline, user can preview the exact properties during creation. (Part 
%of the work)
%\end{enumerate}
%更新绘图区的UI,让用户获得更佳的交互体验。
Update the UI in the canvas area to enhance the interaction experience.
\begin{itemize}
%\item 更新ZOOM/3DZOOM, PAN/3DPAN命令的Cursor Badge。
%\item 创建Line/MLine/Pline/Spline时候,做到所见即所得。
%\item 测试CommandPreview功能对性能的影响。
\item User sees refreshed cursor badge/replacement when using ZOOM/3DZOOM command.
\item User sees cursor been replaced when using PAN/3DPAN command.
\item CommandPreview Benchmark.
\item When creating a Line/Mline/Pline/Spline, user can preview the exact properties during creation.
\end{itemize}
}

\vspace*{0.3\baselineskip}

\cventry{2011.4-2012.11}{AutoCAD - Visual Style}{C++, WPF}{}{}{
%\begin{enumerate}
%\item Maintain the predefined visual styles, such as adding new visual styles and updating the property values.
%\item Maintain the visual style swatchs on VSM.
%\item Add new ribbon controls related to visual styles.
%\item Add new controls for the visual styles on the tool palette.
%\item Fix bugs related to VSM, roundtrip and per-obj visual style.
%\end{enumerate}
Visual Style controls the visual effects of the entities in AutoCAD. There as some predefined styles such as Wireframe, Realistic, etc.
\begin{itemize}
\item Maintain the predefined visual styles, such as adding new visual styles and updating the property values.
\item Maintain the visual style swatchs on visual styles manager (VSM).
\item Fix bugs related to VSM, roundtrip and per-obj visual style.
\end{itemize}
}

\vspace*{0.3\baselineskip}

\cventry{2006.4-2010.11}{AutoCAD MEP - Tool based UI}{C++, MFC, COM/ATL}{}{}
{The feature is to enhance the user interface of the application. Lots of traditional technologies in Windows platform are used, such as COM/ATL, MFC, and ActiveX. I enjoyed the component technology in this product greatly.}

\vspace*{0.3\baselineskip}

\cventry{2003.10-2004.3}{Network Manamement System N2000}{C++, ACE, SQL Server}{}{}{Maintain the background VLAN module of the product N2000. The background modules use ACE as the base framework. The source code is cross-platform, which can be compiled and linked in both Windows and Solaris.}

\section{Education}
\cventry{2002-2005}{Computer Science and Technology}{Wuhan University of Technology}{}{}{}
\cventry{1993-1997}{Building Engineering}{Shantou University}{}{}{}

\section{Skills}
%\textcolor{color1}
\cvcomputer{Language}{C/C++, Python, Objective C, C\#}
           {Unit Test}{GoogleTest, Cpputest}
\cvcomputer{SCM}{Perforce, Git}
           {Editor}{XCode, VS, Sublime Text}
\cvcomputer{OS}{Windows, Mac OS X, Ubuntu}
           {Other}{Scrum, CMake, Markdown, \LaTeX{}}
\cvcomputer{English}{CET-6}{}{}
\end{document}
