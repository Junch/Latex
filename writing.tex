\documentclass{ctexart}
\usepackage[paperwidth=9cm, paperheight=12cm, top=0.2cm, bottom=0.2cm, left=0.0cm, right=0.0cm]{geometry}
\special{papersize=9cm,12cm}
\usepackage[colorlinks,
            linkcolor=black,
            anchorcolor=black,
            citecolor=black]{hyperref}

\begin{document}
\kaishu
\CTEXsetup[name={第, 期}]{section}

\title {\large {\heiti 康奈尔高效写作社团订阅期刊}}
\author {睿秋 \and 剑波}
\maketitle

在2013年的三月,在偶然的打开的一封Lehigh University Daily Event(理海每日要闻)邮件里,看到了Cornell Productive Writing Community(康奈尔高效能写作社团)发来的招收会员的通知,当机立断报名加入。之后每个星期会收到一封来自CPWC的文章,每次都会被严苛的规定、写作者自带神技能的设定、闻所未闻的方法震撼。——原来,英文世界里的写作者是这样一群跟自己较劲的人啊。当时即萌生了将英文订阅期刊翻译成中文的想法,让中文写作者们也从中汲取一些有用的信息。

一共12期专题,非常遗憾,由于邮箱整理不当,我只收集了其中的9期。现在康奈尔研究生学院官网上的此专题链接已经过期,但是非常幸运,遇到剑波,帮我搜到了丢失的3期英文原文。

翻译计划:由于白天工作比较繁忙,每两天翻译并更新一份新的内容。2014年到来之前,力争翻译完全部12qi。:)

这里要感谢Cornell 和 Lehigh百年姊妹校的密切联系,使得这样的活动通知在Lehigh每日推送邮件中循环出现,我才没有错过这个有意思的组织,开启了一段意料之外的写作体验。

是为序。
\newpage

\tableofcontents % Include a table of contents
\newpage % Begins the essay on a new page instead of on the same page as the table of contents 

\section{每天写}
这个tip适用于每个人:每天写。不管写些什么,只要每天坚持。

我知道说起来容易执行起来难。这里我们具体说明一下要如何实践这个tip,包括如何让它带给你高效能的体验。

1. 承诺每天至少写作90分钟

为什么是90分钟呢?

无论生理上还是心理上,这几乎都是你不间断工作的极限了。请保持写作状态90分钟,期间不要离开你的座椅。严格地,不要有间断,不要被打扰。尤其注意,不要在你写作的时候查收邮件或者任何社交网络。(我保证,你可以做到的!)

2. 整整两周时间坚持每天写作

对于我们大多数人来说,两周足够养成一个写作习惯了。我保证,如果你做到了这条,你会发现你写作已经变成一个比以前效率高很多的写作者了。从今天开始,在你写之前和之后进行字数统计(Word里有计数工具)。试试吧。

3. 假如你今天没时间写作,怎么办?

如果正巧遇到放假活动安排,时间冲突怎么办?或者你病恙在身无法写作?又或者你无法在这一天的24小时里腾出完整的90分钟来写作?这样的话,你就必须每天写够15分钟。无论你有多忙,病得多严重,必须写够15分钟。下面分析一下为什么设定15分钟:

{\heiti 对于我们中的一些人来说,写作最难的部分就是开始动笔。}

我们致力于拖延——几分钟,几小时,甚至几天。(据说有人可以拖延几周,甚至几年,当然杰出的、高产的作家们不在此列。)

有时我们拖延,仅仅是因为我们怕没的写,我们怕写出来的东西太烂,我们怕为了写出好的作品在写作过程中遭受种种痛苦。

往往是,当我们意识到,真正的痛苦是如果我们不写,我们可能会失去伙伴、导师,无法完成学位,违约于出版社,失去工作等等,而这些都比写作本身的痛苦大得多,我们才开始动笔。

一旦动笔,你就已经克服了写作中最大的难关。我从没见过,一个人计划写作15分钟会心里盘算30分钟或者60分钟后休息。妙就妙在,你告诉你自己你必须完成的只是15分钟而已,你一定可以确保完成这么短的时间长度的。

一旦你开始写,焦虑就消失了,你会写得比预定时间长的。:)

{\heiti 坚持每天写作可以有效而持续地为你提供高效写作需要的灵感。}

当你写作的时候,你的大脑运转是不同于平时的。我们可以单单想着我们要每天写作这件事,但是确实要动笔会产生有效写作灵感,这是不同于仅仅想着要做这件事的。当你写作的时候,你的那些灵感会发展延伸,你的项目会不断推进……即使当你写着内容杂乱的初稿。

4. 使用提醒或者标志来帮助你达成每天写作。

Anne Lamott (Bird by Bird, 1994一书的作者) 建议:在你的电脑旁边放一个1寸*1寸的相框。每天要写够填满这个相框的内容。我保证你用这个办法就可以顺利完成学术论文了。(译者按:这里,每天只填满那么小的相框的内容的话,要很多天才能完成一篇论文。)当然,如果你使用8寸*12寸的相框,你会更快完成。一切的前提是,你必须每天每天写,那个相框就是一个提醒。

还有一个办法就是发布你学术论文的谢辞或者你的书的致谢页。你要做的就是完成你的作品,只有这样,你生命中那些重要而美好的人们的名字才能变成印刷体。

那么,请这个月履行坚持每天写作吧。

如果开始动笔对你来说从来不是个问题,如果你从不拖延写作计划,如果你从来没有在电脑前坐了几个小时却没什么成果展示的窘状,那么你不需要尝试这个tip。但是,如果你今天还没有写够90分钟(或者15分钟),那么请立即行动吧。

你能做到!

\section{写作是早晨第一件事}

规定写作时间段

上一期中,我鼓励大家养成每天写作90分钟的习惯,每天都不能偷懒。

今天,要跟告诉大家一个新的提高效率的方法——把写作定为每天早上的第一件必办事项。这样,你更有可能坚持执行每天写的计划。

为什么?

因为清晨的时光更不容易被干扰。由于你早上已经完成了一些写作任务,那么,即使后面生活琐事影响到你的写作进度,罪恶感和焦虑也会相应的缓解。而且,如果你把写作定为早上的第一件要事的话,那么你就不必花太多时间让自己进入写作状态,不需要从做一件事的状态转化到写作的状态。

即使你不是一个晨型人,此条对你依然奏效。不信,你试一个星期,尤其在你最忙的那几天,更要贯彻把写作放在第一件事的位置哦。此条适用于各种类型的写作者……尤其是拖延症患者(咱们不能放弃治疗)。

\section{放心大胆写,初稿不怕烂}

我希望各位每天都在坚持写作90分钟……并创作了一大批不堪入目的稿子们:)如果你的文稿并不糟糕,那么小心啦,与你一起写作的小伙伴估计要记恨你了。知名作家Lamott曾表达过她对一位圈内好友的厌烦之情:她连初稿都写得华丽优雅无可挑剔!Lamott甚至确信上帝也会原谅她这种厌恶心理,“当我对我的朋友Tom谈及此事时,他说,在这个问题上,你可以假想一个你创造的上帝,与你同仇敌忾。”

仔细想想,高效能写作的一大障碍就是担心自己写的东西不够好。这种担忧是很常见的,但是请想一下,一份糟糕的初稿是通向优秀作品的必经之路,因为,如果你想不断修改完善你的作品直至符合你的高要求,那么你必须先得有第一稿才行啊。对于第一稿,尽你所能快速写完,不要写写停停、追求完美。《Sides》一书中这么描述的:“如果你想一边写一边修改,那么你两样都做不好。”

Lamott在她书里“狗屁不通第一稿”一章中曾坦白,每次她开始写一个新的作品的时候,她都会惶恐不安。每次她都是在充分的心理建设“我只是要写一个超级无敌烂的草稿而已”之后,才能开始动笔。

告诉自己,其实没人会看你的初稿。初稿完成之后,你会把它改成内容充实、语言流畅的作品。在修改的过程中,你会加强说理的逻辑和论据;你会检查段落结构和过渡连接;你会斟词酌句、旁征博引,使得你的文章超凡脱俗、精妙别致;而且,你还会反复通读几遍,修改病句和错别字什么的。

但是,修改这件事不是现在做,不是此刻你要考虑的。现在你要做的,就是写一篇烂草稿。尤其对于我们中部分追(chui)求(mao)完(qiu)美(ci)的写作者,此条对于我们高效能的写作目标真心管用:写篇烂初稿呗!

亲爱的写作者,今天写了没?我衷心祝愿你今天文思井喷,并创作了一个糟糕的初稿。

\section{避免干扰}
想变得更加高产?规避干扰。

一项针对IT从业人员和在校大学生的研究表明(此人群每天大部分时间在使用计算机),人们工作中每隔12分钟就会被打断一次。因此,如果我们能找到一个规避掉干扰和分心之事的空间来进行写作,那么你会变得更高产。能不能关起门来写作呢?或者,在你的工作区域贴张告示,“正在写论文,稍后再来”?(这条建议来自Kearns和Gardiner)当Stephen King (史蒂芬·金)开始写他的第一部小说的时候,他甘愿屈身于他的房车洗衣间里创作,只为避免干扰。你,一样,也可以通过减少干扰和分心来提高产出。

不过还请特别注意这些每12分钟出现一次的干扰中,有一半是自己分心,即我们转移自己的注意力去查邮件、接电话或者吃零食。有哪些活动出现在影响你更高效能写作的清单上?我们中断自己工作的最普遍的方式是:

* 喘口气休息,紧接着转移了注意力

* 当我们“向极小的外界刺激做出反应,即使并不必立即做出应激反应”

* 接收到去做其他任务的提醒

* 停下来处理工作中的阻碍

* 并从核心任务偏离到其他的、并不相干的任务。(Jin和Dabbish, 2009)

这些干扰我们的事情,不可否认可以减轻压力,增强心理应激能力,以及创造成就感。比如,“小提醒避免了粗心忘事,完成了小任务”。停下你的写作,去网上付个账单,或者发送一封生日祝福邮件,可以让你划掉任务清单上的一项任务。但是,这些事同时降低了你的写作产出,因为这些转移注意力的行为中止了你的思考和写作的连贯性。因为占用了时间,尤其是当你要重新集中注意力、回到你原本手头儿的项目(比如你正在写的东西)上所花的不容忽视的大段时间。

因此,我(再一次地)鼓励大家开展90分钟不离座位的(或者说不查邮件不上社交网站的)写作。写满90分钟,全神贯注。试试看。这一个行动上的改变对你高效能写作的促进效果,会震撼到你。

今天你写了没?

\section{克服完美主义情结}
(翻译:睿秋、剑波。睿秋按:此期前半段重点面向广大硕士和博士研究生群体,指导论文和项目报告的写作方式。)

你是否在尝试写一篇完美的论文或者专题报告?你是否因为觉得自己会写的不完美而迟迟不肯动笔?你是否因为它还不是最好的而不断推迟完成它的时间?你是否等待完美的时间、完美的环境以及完美的心态再去写作?你是否一定要等到灵感爆发时才开始动笔?

借口!去写吧。因为完美是不可能达成的。如果你真的可以写出一篇完整而无可挑剔的论文或专题报告,那么你接下来写什么?你每次写新的都要转换研究方向,因为你已经在这个方向上没什么可做了(如果真是这样,我很怀疑你今后的写作与学术生涯)。

为什么我们中的一些人会是完美主义者?

·过分看重论文或写作项目。那不过是一个领奖学金的工作,并不是你的人生(也许看起来像)。一篇论文很重要,它预示着今后成功并且及时地完成学位,成为博士后或学校的教员,开始从事大量的工作以及获得作为学者的名誉。是的,论文可能是当前需要研究的最重要的学问,但是博士论文或者其他的写作项目并不是你的人生,它不能定义你的全部,专业上讲,也许会证明你的一部分才能。但人生要比你的论文和学术生涯更为重要。(确保在你的人生中,还会有一些人如此这般地定期提醒你此事)。所以不要变得过分看重任何一个你不能完成的,或者从未开始的写作项目。

·没有明确完成一个写作项目所需要的时间。是的,你可以用三年完成你的第一章或者用十年去写本书,但不该发生在毕业获取学位的倒计时开始滴答滴答,或者成为终身教职人员和晋升的进入日程的时候。很多时候,完成该写的内容,并将其公布于世,要比执拗于完成一个无可挑剔、的完美的文章要重要得多。

·来自顾问、导师或者(对的部门在职教员来说的)学院领导的不合适的指导。明确你的预期目标。通过论文开题报告的必要性是什么?为了让你的导师认可你的第四章内容?为了迎合教职评定委员会的标准?达到这些预期目标,甚至超过预期,但并没有人要求你的工作必须是完美无缺的(如果真有人这么干,请联系我,我正要统计一下)。

·我们总有种误解,认为只有等到有灵感时写出的文章才是真正完美的。Silvia (2007)描述说等待灵感才去写对于高效写作来说是最为可笑和讽刺的障碍。如果你认为你只有才感觉到它时再去写,请问问你自己“这样的策略到现在为止实施得如何啊?你对你写作的数量感到满意吗?” 一个成功而专业的作家多产的原因是他们规律地去写作,通常是每天一次。正如Keyes (2003) 所指出的,“严肃的作家写作时,不在乎是否有灵感。随着时间的发展,他们发现有规律的写作是一个比灵感更好的朋友。(p. 49)” (Silvia, p. 27)

你是否如此?

下面Sternberg (1981)所形容的是你吗?“完美论文的神话给研究生们带来了很多问题,从来没有一篇论文,或者同档次的文章、书籍是完美的或绝对完整的。一个成功的在读博士生和图书作家可以在他们学术论文答辩后或者图书出版之后的几天里,想出10处他们还想改进的地方。但是这10处改过之后,博士论文将会被更新。但我经常怀疑这种更新,这么说吧,新旧两版博士论文,新一版中的修改并不会使它变好,改动都微不足道。不禁想起,在《瘟疫》一书中Camus这个角色,他花了他的一生来重写他的小说中的第一句——无数个版本的 马群跑过香榭丽舍大道”。

另外,你成为完美主义者这件事实际上对你想写出完美的作品意义并不大,倒是与你的拖延症状息息相关。Luey(2004)在提到作家的瓶颈时这样说道,“大家所谓的作家的瓶颈,其实是微小的知识分子的瞎矫情和行文逻辑顺序混乱罢了。其中的一个原因就是没有能力阻止对细节的大惊小怪。”你并没有产出很多新的内容是因为你认为总是有很多的缺点,使得你不停地修改他们。“这是拖延症的一种形式,相比创造新的内容,修改已有的要容易的多。请与惰性做斗争”(p. 137-138)

如果你今天还没有写出什么说得过去(但比完美还差一点)内容,那么设定一个500字至800字的目标,开始写吧。

你能做到!

\section{​设定写作deadline}
\subsection{Deadline是第一生产力}

你什么时候最高产?无论何时问我的同事们这个问题,收到的最多的答案是:“我在去度假之前最高效”,或者“去度长周末之前的最后几个小时”,又或者“动身去参加会议之前”。

为什么这会成为大家最高产的时刻?杰克·格罗派,训练运动员、学者和公司管理层,最优化他们表现的那位大师曾指出,因为deadline。你必定要走出房门、乘火车、或者在机场花些时间。而你明确对自己来说最重要的几件事,并会在离开前有限的时间里对付这些事。

那,如果你每天都在这种状态下工作会怎么样呢?如果你制定优先事项列表,并设置完成这些事的deadline呢?这种突击式的设定,会让你更高产么?

很少有无限期的项目,但一旦他们看起来像是没有截止日期的,我们通常会缺乏动力、无组织无纪律地纵容自己不遵守每天写作。写学术著作?写本书?写一章200页的?写篇30页的论文?我们知道不必今天完成它。我们拖延经常是因为我们觉得还有太多时间,以后会有更多灵感,或者将来的某个时刻能准备得更好(然而我们从来没开始,从来不够好)。

\subsection{在deadline下工作,而不是压力}

让我强调一下:我并不是要你制造压力或者焦虑……不要给你所经历的添堵了。而且,尽管有些人声称他们在压力下工作更出色,但这类压力和焦虑并不能产出好的写作作品。你也许可以写出不错的文章,还可能完成了工作,但是这种压力并没有对你的好作品做什么贡献。如果没有压力和焦虑的话,你的作品甚至可以更好。当人们说,他们在压力下工作更出色的时候,其实他们在谈论在deadline下工作。

因此,设定你自己的deadline吧。“今天结束之前我要写完500字”,“接下来的一个小时里我要写5页草稿”,“我去上课之前要写完第一章的第一小节”。然后制定你在可用时间范围里的实施进度,来实现你的目标和满足你的deadline(这个方法对于你创作任何写作项目首稿都有显著功效)。

\subsection{制定你的进度}

艾维塔·泽若巴维,The Clockwork Muse一书的作者说,当他在deadline的驱动下工作的时候,他会明确他要写多少,以及按“预期的总长度(总页数)除以每日进度(每天的页数)”来决定他计划写多少天。比如,他估计每天可以写1~2页;他列出每一个章节计划要写的页数;然后他计算出如果每天写1~2页,那么他的手稿一共6章的话需要93天来完成。在日历上,他标记出所有可以用来写作的日子(划掉那些要教课、度假、出差旅行等,无法写作的日子)。接下来,他强制自己在这些天里按计划、按规定的速度来写作。如果他每天都写的话,13周之内他这本书就能写完;而如果他一周只有两天可以写作的话,那么他要花46周才能写完首稿。

在计划和deadline的驱动下写作(而不是一个没有终点的、每天90分钟的要求)会对你有效么?试试看,检验一下这个方法对你高效能写作的益处大不大。它会帮助你减少拖延,避免写作时过度修改润色,并且帮助你按时搞定首稿或者最终稿。

你今天写了么?

\section{暴写暴作}
放纵式写作:别!

你这个学期有没有干过放纵式写作?结果如何?

你曾经是一个放纵式写作者么?我曾是。我现在尽量避免放纵式写作,因为它是低效率、低产出的。但是,坦白从宽的话,回溯到多年前我刚当副教授的时期,我发现不可能在周中抽出很多时间写作。所以我试着周末全天用来写作。当周末也被论文评分、批改试卷、分析数据和写申请报告等事占据,我就会把春假那周或者学期中的两周的假期全部计划为全天写作。结果就是,这样工作得并不好。尝试在大块的时间里、绞尽脑汁地猛写一通,很难真写出多少内容。而且,这样会产生更多的写作压力和焦虑情绪,或者缺少(压力)。

放纵式写作并不高效

放纵式写作的症结是什么?首先,如果你不能规律地写作的话,比如每天90分钟,那么当你在一周、或者一个月、或者一个学期结束,终于找到可以写作的时间的时候,实际上是终于可以开始写的时候,要写出又多又好的作品,你会有更大的压力。我的放纵式写作大概是这样的:我向自己承诺,既然周中啥都没写,那么周六和周日从早上8点写到晚上6点……这10个小时的工作压力巨大。焦虑产生,写作延迟,更多压力来袭,更焦虑,写作继续延迟。实事求是地讲,哪种感性的工作,你能长达10个小时都做得特好?我经常会整理下笔记啊,看会儿书啊,构思一下啊,或者完成一些别的需要注意力的工作——于是10个小时过去了,我真正写出来的东西少得可怜。这种模式进行了几周之后,我可以算一算如果我每天只写15分钟,总共能写多少,和企图每周来一次10小时的长时间写作到底写了多少。

恐惧和厌恶

放纵式写作的另外一个问题就是,如果每一次绞尽脑汁猛写一通之间的间隔时间很长,那回到聚精会神和高效写作状态需要大量的时间。此外,每天写作会让你在较小的压力下变得高产。不规律的写作,“注意力从原本要完成的事务中转移了——容易焦虑的人会因担心产出数量而分神。”反过来,如果注意力集中在每天写作,而少关注产出数量……那么你反而容易达成的你所追求的数量。

而且,在负罪感和焦虑的压迫下,放纵式写作者尝不到写作过程的甜头。因为长时间的“纵欲”,精疲力竭、浑浑噩噩的写作经历,会加深放纵式写作者对写作这件事的厌恶感。

因此,如果你正在暴写暴作,并感到痛苦,就此结束吧,下周开始每天写90分钟。然后看看效果如何。

今天你写了么?

\section{结束写作前的最后五分钟}

这里有一个在每天写作时间的最后五分钟里的务必做的事,可以为你节省时间和提高你的效率。在你停止写作之前的那五分钟里,列出日后继续写作下一个部分的要点和想法。

我最近才注意到这件事,事因是我继续写一份我几周之前没写完的东西。之前我写下的最后一句是:”这里至少有三个方法来解释这一现象。”当时我就此停笔……而结果就是,我想不起来当初本来打算写的三种方法是什么了。当我终于完成这个部分的稿子的时候,我还是不确定后来写的三种方法到底是不是之前确定的那三种。也有可能实际上有六种好方法呢。我们永远无法得知了。

“从前文中回忆”,在我们写一个话题的时候与在我们仅仅是构思的它的时候,是不一样的认知过程。因此,一旦你进入写作的思维路径,你的大脑通常在一种可以把你带到你要去的任何地方的模式中工作。你很难总是知道,当你坐下来写作的时候,你会写出什么。写作的过程将你带到哪儿算哪儿。因此在你计划的写作时间结束的时候,当你必须停下来的时候,写个列表,或者提纲,或者思维导图,记下来你接下来要写的一些想法内容。第二天,当你再开始写作的时候,利用这张清单来帮助你更快速地开始。

正像Csikszentmihalyi在1990年书中所述,为了更加高效能地写作,“思路流”这个概念是要讲一讲的。“手头上正在写的东西会因其复杂性,吸引写作者的注意力,乃至令人全身心地投入其中。”你能否回忆起,某一次你在写作,并进入了“思路流”之中?Csikszentmihalyi指出,这个思路流是伴随着八个状态形成的。尽管,他并没有特意强调对写作影响,但是很明显,这八条正是我们写作时追求的状态:

·目标明确

·反馈是即时的

·机遇与能力之间达到平衡

·全神贯注

·“当下”意义非凡

·感觉自己在控制中

·时间感是可调节的

·经历自我意识的减少

如果你在写作中满足了上述八条中的大多数,那么你的写作过程会有怎样的不同呢?举例来说,当Csikszentmihalyi解释第三条“机遇与能力之间达到平衡”,他写到:“如果我们相信某个任务是可行的,那么我们更容易全身心投入其中。但如果它超出了我们的能力范围,我们对它的反应则会是焦虑……理想的状态可以由一个简单的公式概括:思路流在挑战和能力都很高且相等的情况下发生。

如果你每天写作90分钟,我希望你已经经历了一两次这种思路流发生。在这种思路流中,你已经战胜了恐惧和焦虑,并且有能力写出更好的东西,和持续写作更长时间。我鼓励大家坚持每天写,直到你们感受到这种思路流的美妙——那种全神贯注于你的任务的所有复杂性,完全投入并享受写作过程的感觉——然后保持写作的状态。如果这状态还没发生在你身上,我保证,它会来的。

\section{保持你的动力}
(翻译:睿秋。译者按:本期可以看作前面八期的回顾和整理,接下来还有十、十一、十二,新的内容,更多精彩。)

“我坐在黑暗中,等待一簇光亮略过铅笔尖。”比利·柯林斯。

很诗意对吧,但是并不实际。如果你总是没有灵感,或者准备欠佳呢?先不提效率低下,仅在有灵感和准备就绪的情况下写作,也是不聪明的做法。等待灵光一现并不可取。而采用下述这些激励策略才是正道。

1. 设定一个每日写作的目标。

制定一个在你的预计的写作时间里可以完成的计划,预计的写作时间指我们之前倡导的每周至少5天,每天1-2小时。计划可以类似于:每天两页;每天一千字;为第二章列个提纲,并写出每个段落的标题和副标题;完成第三章两个部分的首稿。当你写作时聚焦在一些具体的目标,并且某一时间段只完成一个目标,你会更加高产。记住,你不可能今天写完一部学术大作或者一本书,但是你今天可以写完三个段落,或者三页……并推动你的写作项目。

2. 把写作设定为每天早上的工作,越早越好。

如果你畏惧或者厌恶写作,那么,当你写完,你会如释重负,这一天剩下的时光就可以尽情去做其他必办事项了……而不必总是惦记着写作这件事。如果你发现某天的可以用来写作的时间非常有限,那就在你通常开始一天的其它活动之前,划出1~2小时来写作,这样一来,你就不必担心你会……又一次地……找不出写作的时间。即使你是一个“夜猫子”,如果你每天早上的第一件事就是写点东西的话,你每天的写作产出会增加,每天也会写得更充实。(嘿,如果你突然发现那天的晚些时候还有更多空余时间,你当然会更高产了。)

3. 思在前,逆向计划。

这一条建议来自Michael Zygmond和Beth Fischer,这两位是来自匹兹堡大学的神经学家。如果你的论文或者演讲有截止日期,或者答辩日期,那么从那一天往前数,规划你的写作安排。举个例子,如果你计划2014年5月16日博士毕业,你就知道你的毕业论文终稿一定要在5月11日之前提交。如果你为自己在答辩之后预留两周时间来完善你的论文,那么你必须在4月27日之前完成答辩。那么你就要在4月13日之前提交你的答辩论文。那么4月9日那周你该做什么?再之前的一周呢……倒序一直计划到现在这周。如果你想按时完成一个长期目标的话,这个“思在前,逆向计划”的方法会帮助你明确你本周必须完成什么,以及你今天必须完成什么。

4. Deadline驱动进程。(详见“第六期:设定写作Deadline”)

如果你发现自己总是错过deadline,甚至你还有总有很好的借口,那么你就应该知道你必须规划设定更多天来写作,或者你必须每天多写几个小时,再或者你必须每个写作时间段里都多写一些。否则,你会错过你的deadline,你将在会议演讲、工作面试时自尝苦果,还有可能毕业延期。(我发现这一系列事件非常能激发潜能,但是同时会带来痛苦。所以有可能的话,激励的策略还是现在就动笔,来避免将来的痛苦。)

5. 写作无好坏,修改见水平

——哥伦比亚大学GSAS教育中心的院长Steve Mintz曾对我们的论文提升团成员如是说。Lamott (1994)曾提到,“写完,你才能整理好。”Shaw(1993)曾说过,“写作能力没有好不好一说,修改能力倒是有优劣。”这句话可以鼓励你,不必今天就写出定稿或者一个优秀的首稿。你只需写完一个首稿,以便你之后编辑的时候把它改成优秀作品。

6. 奖励进步。

我们中的一些人会充分地把完成好的作品带来的满足感当作奖励。(我们有一个内在的行为轨迹)而另一些人需要更多——看得见摸得着的、诱人的、实实在在的奖励。因此,在你写作中的一些时间节点上奖励自己,但一定是写完足够数量之后。(不能写完每句话都来块糖。)Silvia提醒我们“坚决不能把‘不用写了’作为写作的奖励。通过放宽对写作计划的执行来奖励写作进步,就像戒烟取得进步的时候奖励自己吸一根烟……别毁了你的良好写作习惯。”

7. 利用其他优秀作家的故事(以及他们遭受的一切)来激励(并宽慰)自己。

Ralph Keyes在《写作者希望之书》(The Writer's Book of Hope)中解释到,他的一个文件夹里存着这样一些故事:一位“旧金山考官”杂志的编辑退回Rudyard Kipling的文章,并附一张留言,“这里不是业余写作者的幼稚园。很抱歉,Kipling先生,但是你真的并不会用英语这门语言。”即使优秀的作家也有不走运的时光。

8.读别人的致谢语来激励你:“它们会是一个宝典,充满了有用而让人安心的信息。”(Keyes, p. 143)

作者会列举出长长的一个致谢名单,在那一次次AFD(anxiety, frustration and despair焦虑受挫绝望)发生时,这些人“鼓励他们,支持他们,让他们保持良好精神状态”。当我的研究生们意志消沉、动力不足时(甚至对动笔写作产生畏惧心理),我会鼓励他们写一写致谢语。写起来简单有趣,它会让你的学术著作看起来更像一本书,而且,你的导师和评审委员会的成员永远都不会对你的致谢语指手画脚。

9. 这里还有一个很棒的激励策略:一旦你有一天没有写作,就捐出5美元给你喜爱的美国总统候选人的敌对党。(Boice, 1990)

\section{思路卡住了?}
(翻译:睿秋,剑波)

几乎所有的写作者在写作时都体验过思路卡壳或滞住的感觉。那么如果这种情况发生在你身上你会怎样做?这里有一些建议:

·一位同事曾经告诉我他这样建议他的学生的:把你的手指放到键盘上,然后开始打字。强制自己打字,即使你在打的是“研究院的老巫婆要求我每天必须写作90分钟,所以我现在在写作。当然,我现在写的没有任何意义,但是起码我在写作。”没开玩笑。开始写作/打字。想一想你的主题和你想要写的东西。“好吧,我今天的主题是道德推理的哲学发展观,所以我猜我应该写卢俊(Rousseau)、杜尔凯姆(Durkheim)和皮亚杰(Piaget)还有科尔伯格(Kohlberg)。好吧,那么我将从皮亚杰开始。皮亚杰可能是……”你可能会写一些废话,但是最终你会写出优质的内容。或者至少是得体的内容。当你在校订时,它会变得越来越好。

·一份手稿、章、节或者段落最难的部分通常是开头和结尾。所以先从中间开始。一旦你写下了有实质性的内容,你可以去写开头和结尾了。

·下面是帮助我开始写作的方法。如果我正在写一个章节,我首先会写下每一个和这个章节开头到结尾都相关的标题。然后我回过头来,写在每个标题下面的子标题。接着,在每个子标题下,我会写下每个段落的想法或观点。到这时,我可能在每个子标题下已经有5到8个想法和观点。此时,也只有到了此时,我才会挑选一个章节去写完它。按这种方式去做,我已经知道每个章节的脉络走向。这对我有帮助,是因为我发现当我写一个章节时,即使那段章节写的很好,我仍会为之后的章节感到焦虑。“如果我不知道接下来该从哪里入手怎么办?如果我不能想出接下来的章节写什么怎么办?“因为我已经列出了每个章节要写什么,我就不会感到那么焦虑了。而且由于整个手稿或章节已经有许多观点和想法了,我很确定这些对我来说是可以完成的。我能搞定它!

·如果你真的被卡住了,这里有些建议对那些动觉型学习者和写作者(思路跳脱的、好动的人)有效。站起来和活动一下。踱步。去散步或者跑步。但是你必须在你做这些时思考你的主题。这并不是打断你的写作。这是靠运动想出你需要写的东西。(我曾经有个同事,他的办公室正好在学校跑道旁边,当他写一份手稿时,他会围着跑道跑一英里,然后回去写一个章节。然后再起身围着跑道跑一圈,接着写下一章节。另一个同事告诉我说,当她在写明尼苏达大学的时候写学术论文,她把她妈妈餐厅的墙纸全部揭下来了。现在看来,她妈妈显然是想清理掉墙纸的。但是谢莉尔(Cheryl)会写一会儿,思路停滞了,揭一会儿墙纸,同时构思下一段落,然后她会坐下来将这些写出来。)所以,如果你有动觉型学习或写作的风格,尝试一下吧。但是记住,运动、跑步和刮墙纸不是写作。你仍然必须每天至少写90分钟。

·你现在已经写了快一个月了。进展如何?记住,一旦你开始写作,你会在写作的过程中得到涌现的想法。所以每一天不要害怕开始写作。即使当你脑袋里没有任何可以写的,当你开始写——上面建议的无意义录入或者你的第一份蹩脚的草稿——当你思考和写作而不是想将要写什么的时候,你的认知过程将会改变。(我开始写一个新的章节并且我一点也不知道怎样开始写它。即使断断续续地思考了3天(当我写其他东西的时候)也没有任何一点线索或观点。但是一旦我开始打字,我得到只有在我开始写的时候才出现的非常棒的三页前言。所以说真的,就此开始写吧。)

·如果你没有思路了,不要把它称为写作者的瓶颈。“学院派的写作者不会有写作者的瓶颈……你并不是在精巧地创作深度叙事或者组合隐喻来展现人类内心的神秘。你那微妙的、变化的分析不会使读者流泪,尽管它要是沉闷乏味倒可能会使人……写作者的瓶颈和不去写的行为相比算不了什么……对于写作者的瓶颈的治愈方式就是……去写作”(《西尔维娅》2007,44-45页)。“就想外星人只绑架那些相信外星人会绑架人类的人一样,写作者的瓶颈只在那些相信它的人身上奏效。”(47页)

·如果你想要阅读一位作家关于另一个变得更高效的方法,即每天只写两页的建议,请看“每天两页带来的可观的满足感”http://chronicle.com/article/The-Considerable-Satisfaction/22085。不过记住,阅读关于写作的资料不是写作,你仍然一定要去写。

\section{写作时间管理}
如果你在思考是否可以在写作时间里更加高效,那么请读以下这些补充的时间管理小贴士。它们会不会对你有效呢?

1. 试试这个:在你开始新的一章或者新的小节之前,想一想你必须要做的事:你要涉及多少内容?需要包含哪些想法和论点?要用几个段落?然后思考一下在这个主题上……由你来写……好的作品需要多长时间。记下这个时间长度。接下来从那一节开始写,并追踪记录下完成计划的量实际用掉的时间。当你写完的时候,要回过头来看看花了多长时间。(你有没有过实际用时比计划用时短的经历?有的话,恭喜你。这可从来都没在我身上发生过。)总是比计划耗时更长。接下来的一步最关键,也是这个训练的核心部分。分析为什么实际花的时间长。也是是因为你比计划中休息的次数更多,吃了更多的零食,看了不少电视,或者其他的干扰。那么,你知道你该采取什么行动了么?明确并克服这些已被证明为你的弱点的干扰项。(不要连接网络。不要查邮件,上Facebook。关掉手机。一口气写足90分钟。)另一种可能,你是否给自己预计了足够长的时间来完成这个写作任务呢?会不会是你预想中的写作任务比实际简单,因此预计会完成得比较快?会不会是这些想法在你大脑中构思比实际写在纸上或屏幕上更容易,因此你低估了将这些想法转化成文字格式需要的时间?是不是你以为只要90分钟就能完成的写作任务,实际上需要两倍的时间?成为一个差劲的预估时间的人并没有什么大不了,这只意味着你必须采用下述两种方法之一来迎合你的时间规划:一,你必须更早地开始写作项目。比你预计的开始时间提早一周、一个月、甚至更早开始。二,你也可以比以往任何时候都更迅猛地工作。我建议第一种方式。它会使你保持一个从容智者的姿态(先不提对朋友、所爱之人的影响),同时可以提高你的工作质量。

2.如果你被告知要停止写作,或者只在有灵感的时候写作,你能完成多少写作量呢?Boice(1990)在一个由27名教师组成的“写作规劝会”进行了总结汇报。所有与会教师都说出了阻碍他们完成写作的问题,他们都有可操作的写作项目要完成。Boice将教师们分派到三个处境之一。估计最让人羡慕的就是被安排10周不许写作,除非有“紧急任务”的这一组的9个人了。“弃绝组”的几位设想停止写作10周,会让他们进步更多,产生更多写作创意和想法。Boice告诉第二组的9个人,在10周里安排50次写作时间,但是只在他们情绪到位的时候进行。(这也是个不错的任务,对么?)这些“随性组”写作者们也预想他们会有更多有新意的写作思路。Boice要求剩下的9名教师在10周里必须安排50次写作时间。但是,如果他们没有在计划时段里写满至少3页,或者最终没有写满150页,一张被他们自己签字生效的支票将会寄往一个他们讨厌的组织。(民主党?共和党?美国步枪协会?计划生育委员会?)这个“昂贵应急预案组”的教师们,毫无疑问抽中了下下签,他们预测可能会高产,但绝对不会写出什么新意。你知道后来发生了什么?“强制执行”组写出的数量是“随性组”的三倍,同时是“弃绝组”的十五倍。这些教师的自我反馈中,第三组,也就是强制写作不然就要付支票的那组,每个写作日都有“有用的、新颖的想法”;而如此反馈的教师占该组教师总数的比例在第二组只有一半,在第一组只有五分之一。一位强迫去写作的教师说到:“实际情况与我想象的完全不同。我并没有感到特别有压力,因为我甚至都很少去想压力这个事情……如此自律使人感觉非常棒。而且最喜欢的部分,实际是,开始动笔变得如此容易。没有挣扎。我期待开始。我一整天都会想着我打算写点什么。有时候,是我甚至企图早点开始动笔。这的的确确听起来不像我了。”(大笑)(Boice, 1990,p.81)

3. 来自Boice的另外一个建议是,如果你每天开始动笔都有困难的话,这个建议也许会看起来违反常理:为写作设定限制。“在你感觉一切就绪之前,开始动笔写。在你感觉你满意了之前,结束写作。”(Boice,1990,p.86)Boice解释说,不要等到你有了完美的计划和在脑子里构思了完美的草稿,当你还没无可挑剔地写完它的时候,就把它拿去检查和修改吧。“要讲解一下放弃某种控制欲的好处了,比如,追求完美,或者,稍微健康一点的控制欲,比如,在没有不必要的焦虑情况下,有能力进行舒适的工作和交流。”(p.87)Boice提供了一种有意思的对抗拖延的方法——这是一种限制的设定。当你拖延的时候,你限制自己只可以来一次在有限的几天里的猛烈而短暂的突击式写作,代替日复一日的写作。“拖延症写作者往往忽略的一点是,那些最后关头的突击写作带来的令人反感的属性——夹杂在心率交瘁、焦躁、丧失对写作能力的信心等情绪中。”(p.87)上文提到了一些更好的方式来设定限制,而它们会带来更高效的写作产出。

\section{如何像作家一样思考和行动}
很简单,你必须去写。你也会思考,阅读,焦虑,记笔记,受折磨,组织你的素材和焦虑,你会买一个新桌子,焦虑,你会削尖你的铅笔,受折磨,你会洗你的盘子,吃点零食……但这都不是写作。你必须去写点东西。

而且你必须想要去写。不是为了去完成,或发表,或得到一份工作,或接受赞美,喜爱和认同,而是为了去写作。“我今天真的很想去写作并且我为它创造所有的机会,而不是逃避所有的机会。我今天在写作是因为我想要去写。”所以所有你需要做的就是去写作。你不必去完成它。你仅仅需要去写作。时光在流逝。不要等到你准备好了再去写。不要等到其他事情都完成了,不要等到你充分休息好了,也不要等到你读完关于你要写的主题的每一本书和文章。你永远不会准备好这些或者完成这些。这不会发生。如果你等待,你就永远不会去写,永远不会成为一个作家。作家真的去写。

写作是艰难的。如果你发觉写作对你来说是个挑战,并不是你的问题。即使是最杰出的作家也说写作很艰难,很难。在一篇关于兰登书屋的主编罗伯特·路米斯的文章中,蒂尼提亚·史密斯这样描述路米斯给作者们的“友好的建议”。罗伯特这样对吉米·雷诺说,“无以伦比啊!吉米你做到了。这本小说棒呆了!”稍作停顿,“大体上是的,除了开头和结尾。”他还这样回复米娅·安格鲁,“真的很不错——基本上吧。”对凯文·特瑞林,他说,“基本就是它了,一切都很好,除了开头和结尾。”而这些,毫无疑问,也要求中间部分完全重新写。

\begin{itemize}
\item 达成每天至少写作90分钟。(译者按:接下来都是老生常谈,不再详述。)
\item 融入写作的思路流中。一直写下去,知道你进入了美妙的思路畅通的状态中。
\item 好的作品要花很长时间,你必须早些开始,并坚决不给拖延留机会。
\item 做好心理准备。当你写的越来越多,你会成为一个更优秀的写作者,写草稿,甚至编辑修改,都会变得更高效。当你开始像作家一样思考和行动,渐渐地你也可以从编辑的角度去思考了。但是,很有可能,一切都并没有变得容易完成。对这点,我表示遗憾,但是我坚信你可以做到!
\end{itemize}

\end{document}